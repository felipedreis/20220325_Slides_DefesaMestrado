


\section{Introdução}
%-- 00
\subsection{Contexto}
\begin{frame}\justifying
	\begin{itemize}
	\end{itemize}
\end{frame}

\subsection{Objetivos gerais e específicos}
%-- 05
\begin{frame}\justifying
	\newtheorem{mur}{Objetivo Geral}
	\begin{mur}[]
		\justifying
		\normalsize{}\newline
	\end{mur}
\end{frame}

%-- 06
\begin{frame}\justifying
	\newtheorem{mur2}{Objetivos Específicos}
	\begin{mur2}[]
		\justifying
		
	\end{mur2}
\end{frame}




% O objetivo geral deste trabalho é consolidar a arquitetura D-Optimas do ponto de vista de um sistema distribuído, tolerante a falhas, com balanceamento de carga e transparência de localidade, tornando-a resiliente e escalável horizontalmente. Com isso será possível executar simulações com problemas de larga escala em um \textit{cluster} com uma variedade de agentes e estratégias de busca. Espera-se observar a adaptação da arquitetura ao problema, com os agentes colaborando e, dessa forma, verificar, neste comportamento,  o surgimento de uma hibridização dinâmica das meta-heurísticas.
